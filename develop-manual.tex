\documentclass[12pt]{book}

\usepackage{xeCJK}
\usepackage{listings}
\usepackage{multirow}
\usepackage{longtable}
\usepackage{fontspec}
\usepackage{ctex}
\usepackage[top=3cm,bottom=3cm,left=3cm,right=3cm, headsep=10pt,a4paper]{geometry} % Page margins
\usepackage{graphicx} % Required for including pictures
\graphicspath{{Pictures/}{Image/common/}} % Specifies the directory where pictures are stored
\usepackage{lipsum} % Inserts dummy text
\usepackage[ddmmyyyy]{datetime}
\usepackage{tikz} % Required for drawing custom shapes
\usepackage[english]{babel} % English language/hyphenation
\usepackage{enumitem} % Customize lists
\setlist{nolistsep} % Reduce spacing between bullet points and numbered lists
\usepackage{booktabs} % Required for nicer horizontal rules in tables
\usepackage{xcolor} % Required for specifying colors by name
\definecolor{ocre}{RGB}{243,102,25} % Define the orange color used for highlighting throughout the book
\usepackage{avant} % Use the Avantgarde font for headings
%\usepackage{times} % Use the Times font for headings
\usepackage{mathptmx} % Use the Adobe Times Roman as the default text font together with math symbols from the Sym­bol, Chancery and Com­puter Modern fonts
\usepackage{microtype} % Slightly tweak font spacing for aesthetics
\usepackage[T1]{fontenc}
\usepackage[utf8]{inputenc}
{\renewcommand{\bibname}{References}}
\usepackage[backend=bibtex,style=numeric]{biblatex}

\defbibheading{bibempty}{}

\usepackage{calc} % 简单的计算功能
\usepackage{makeidx} % 创建索引
\makeindex % Tells LaTeX to create the files required for indexing
\usepackage{titletoc} % 目录操作
\usepackage[bookmarksopen,bookmarksdepth=3]{hyperref}


%When compile under liunx% 

%\setCJKmainfont{WenQuanYi Micro Hei} % 设置缺省中文字体

\contentsmargin{0cm} % Removes the default margin

\lstdefinelanguage{JavaScript}{
	keywords={typeof, new, true, false, catch, function, return, null, catch, switch, var, if, in, while, do, else, case, break},
	keywordstyle=\color{blue}\bfseries,
	ndkeywords={class, export, boolean, throw, implements, import, this},
	ndkeywordstyle=\color{darkgray}\bfseries,
	identifierstyle=\color{black},
	sensitive=false,
	comment=[l]{//},
	morecomment=[s]{/*}{*/},
	commentstyle=\color{purple}\ttfamily,
	stringstyle=\color{red}\ttfamily,
	morestring=[b]',
	morestring=[b]"
}

%Figure显示为中文的“图”
\renewcommand\figurename{图}

\setcounter{tocdepth}{2}

%Set Code Format%
\lstloadlanguages{C, csh, make,python,Java,JavaScript}
\lstset{	  
	alsolanguage= XML,  
	tabsize=4, %  
	frame=shadowbox, %把代码用带有阴影的框圈起来  
	commentstyle=\color{red!50!green!50!blue!50},%浅灰色的注释
	frameround=tttt,  
	rulesepcolor=\color{red!20!green!20!blue!20},%代码块边框为淡青色  
	keywordstyle=\color{blue!90}\bfseries, %代码关键字的颜色为蓝色,粗体  
	showstringspaces=false,%不显示代码字符串中间的空格标记  
	stringstyle=\ttfamily, % 代码字符串的特殊格式  
	keepspaces=true, %  
	breakindent=22pt, % 
	breaklines=true,%设置代码自动换行 
	numbers=left,%左侧显示行号 往左靠,还可以为right,或none,即不加行号  
	stepnumber=1,%若设置为2,则显示行号为1,3,5,即stepnumber为公差,默认stepnumber=1  
	%numberstyle=\tiny, %行号字体用小号  
	numberstyle={\color[RGB]{0,192,192}\tiny},%设置行号的大小,大小有tiny,scriptsize,footnotesize,small,normalsize,large等  
	numbersep=8pt,  %设置行号与代码的距离,默认是5pt  
	basicstyle=\ttfamily, % 这句设置代码的大小  
	showspaces=false, % 
	escapechar=`,
	flexiblecolumns=true, %  
	breaklines=true, %对过长的代码自动换行  
	breakautoindent=true,%  
	breakindent=4em, %  	   
	aboveskip=1em, %代码块边框  
	tabsize=4,  
	showstringspaces=false, %不显示字符串中的空格  
	backgroundcolor=\color[RGB]{245,245,244},   %代码背景色  
	%backgroundcolor=\color[rgb]{0.91,0.91,0.91}    %添加背景色  
	escapeinside={``}{\_},  %在``里显示中文  
	%% added by http://bbs.ctex.org/viewthread.php?tid=53451  
	fontadjust,  
	captionpos=t,  
	framextopmargin=2pt,
	framexbottommargin=2pt,
	abovecaptionskip=-3pt,
	belowcaptionskip=3pt,  
	xleftmargin=4em,
	xrightmargin=4em, % 设定listing左右的空白  
	texcl=true
}

% Part text styling
\titlecontents{part}[0cm]{\addvspace{20pt}\centering\large\sffamily}{}{}{}

% Chapter text styling
\titlecontents{chapter}[1.25cm] % Indentation
{\addvspace{12pt}\large\sffamily\bfseries} % Spacing and font options for chapters
{\color{ocre!60}\contentslabel[\Large\thecontentslabel]{1.25cm}\color{ocre}} % Chapter number
{\color{ocre}}  
{\color{ocre!60}\normalsize\;\titlerule*[.5pc]{.}\;\thecontentspage} % Page number

% Section text styling
\titlecontents{section}[1.25cm] % Indentation
{\addvspace{3pt}\sffamily\bfseries} % Spacing and font options for sections
{\contentslabel[\thecontentslabel]{1.25cm}} % Section number
{}
{\hfill\color{black}\thecontentspage} % Page number
[]

% Subsection text styling
\titlecontents{subsection}[1.25cm] % Indentation
{\addvspace{1pt}\sffamily\small} % Spacing and font options for subsections
{\contentslabel[\thecontentslabel]{1.25cm}} % Subsection number
{}
{\ \titlerule*[.5pc]{.}\;\thecontentspage} % Page number
[]

% List of figures
\titlecontents{figure}[0em]
{\addvspace{-5pt}\sffamily}
{\thecontentslabel\hspace*{1em}}
{}
{\ \titlerule*[.5pc]{.}\;\thecontentspage}
[]

% List of tables
\titlecontents{table}[0em]
{\addvspace{-5pt}\sffamily}
{\thecontentslabel\hspace*{1em}}
{}
{\ \titlerule*[.5pc]{.}\;\thecontentspage}
[]

%----------------------------------------------------------------------------------------
%	MINI TABLE OF CONTENTS IN PART HEADS
%----------------------------------------------------------------------------------------

% Chapter text styling
\titlecontents{lchapter}[0em] % Indenting
{\addvspace{15pt}\large\sffamily\bfseries} % Spacing and font options for chapters
{\color{ocre}\contentslabel[\Large\thecontentslabel]{1.25cm}\color{ocre}} % Chapter number
{}  
{\color{ocre}\normalsize\sffamily\bfseries\;\titlerule*[.5pc]{.}\;\thecontentspage} % Page number

% Section text styling
\titlecontents{lsection}[0em] % Indenting
{\sffamily\small} % Spacing and font options for sections
{\contentslabel[\thecontentslabel]{1.25cm}} % Section number
{}
{}

% Subsection text styling
\titlecontents{lsubsection}[.5em] % Indentation
{\normalfont\footnotesize\sffamily} % Font settings
{}
{}
{}

%----------------------------------------------------------------------------------------
%	PAGE HEADERS
%----------------------------------------------------------------------------------------

\usepackage{fancyhdr} % Required for header and footer configuration

\pagestyle{fancy}
\renewcommand{\chaptermark}[1]{\markboth{\sffamily\normalsize\bfseries\chaptername\ \thechapter.\ #1}{}} % Chapter text font settings
\renewcommand{\sectionmark}[1]{\markright{\sffamily\normalsize\thesection\hspace{5pt}#1}{}} % Section text font settings
\fancyhf{} \fancyhead[LE,RO]{\sffamily\normalsize\thepage} % Font setting for the page number in the header
\fancyhead[LO]{\rightmark} % Print the nearest section name on the left side of odd pages
\fancyhead[RE]{\leftmark} % Print the current chapter name on the right side of even pages
\renewcommand{\headrulewidth}{0.5pt} % Width of the rule under the header
\addtolength{\headheight}{2.5pt} % Increase the spacing around the header slightly
\renewcommand{\footrulewidth}{0pt} % Removes the rule in the footer
\fancypagestyle{plain}{\fancyhead{}\renewcommand{\headrulewidth}{0pt}} % Style for when a plain pagestyle is specified

% Removes the header from odd empty pages at the end of chapters
\makeatletter
\renewcommand{\cleardoublepage}{
	\clearpage\ifodd\c@page\else
	\hbox{}
	\vspace*{\fill}
	\thispagestyle{empty}
	\newpage
	\fi}

%----------------------------------------------------------------------------------------
%	THEOREM STYLES
%----------------------------------------------------------------------------------------

\usepackage{amsmath,amsfonts,amssymb,amsthm} % For math equations, theorems, symbols, etc

\newcommand{\intoo}[2]{\mathopen{]}#1\,;#2\mathclose{[}}
\newcommand{\ud}{\mathop{\mathrm{{}d}}\mathopen{}}
\newcommand{\intff}[2]{\mathopen{[}#1\,;#2\mathclose{]}}
\newtheorem{notation}{Notation}[chapter]

% Boxed/framed environments
\newtheoremstyle{ocrenumbox}% % Theorem style name
{0pt}% Space above
{0pt}% Space below
{\normalfont}% % Body font
{}% Indent amount
{\small\bf\sffamily\color{ocre}}% % Theorem head font
{\;}% Punctuation after theorem head
{0.25em}% Space after theorem head
{\small\sffamily\color{ocre}\thmname{#1}\nobreakspace\thmnumber{\@ifnotempty{#1}{}\@upn{#2}}% Theorem text (e.g. Theorem 2.1)
	\thmnote{\nobreakspace\the\thm@notefont\sffamily\bfseries\color{black}---\nobreakspace#3.}} % Optional theorem note
\renewcommand{\qedsymbol}{$\blacksquare$}% Optional qed square

\newtheoremstyle{blacknumex}% Theorem style name
{5pt}% Space above
{5pt}% Space below
{\normalfont}% Body font
{} % Indent amount
{\small\bf\sffamily}% Theorem head font
{\;}% Punctuation after theorem head
{0.25em}% Space after theorem head
{\small\sffamily{\tiny\ensuremath{\blacksquare}}\nobreakspace\thmname{#1}\nobreakspace\thmnumber{\@ifnotempty{#1}{}\@upn{#2}}% Theorem text (e.g. Theorem 2.1)
	\thmnote{\nobreakspace\the\thm@notefont\sffamily\bfseries---\nobreakspace#3.}}% Optional theorem note

\newtheoremstyle{blacknumbox} % Theorem style name
{0pt}% Space above
{0pt}% Space below
{\normalfont}% Body font
{}% Indent amount
{\small\bf\sffamily}% Theorem head font
{\;}% Punctuation after theorem head
{0.25em}% Space after theorem head
{\small\sffamily\thmname{#1}\nobreakspace\thmnumber{\@ifnotempty{#1}{}\@upn{#2}}% Theorem text (e.g. Theorem 2.1)
	\thmnote{\nobreakspace\the\thm@notefont\sffamily\bfseries---\nobreakspace#3.}}% Optional theorem note

% Non-boxed/non-framed environments
\newtheoremstyle{ocrenum}% % Theorem style name
{5pt}% Space above
{5pt}% Space below
{\normalfont}% % Body font
{}% Indent amount
{\small\bf\sffamily\color{ocre}}% % Theorem head font
{\;}% Punctuation after theorem head
{0.25em}% Space after theorem head
{\small\sffamily\color{ocre}\thmname{#1}\nobreakspace\thmnumber{\@ifnotempty{#1}{}\@upn{#2}}% Theorem text (e.g. Theorem 2.1)
	\thmnote{\nobreakspace\the\thm@notefont\sffamily\bfseries\color{black}---\nobreakspace#3.}} % Optional theorem note
\renewcommand{\qedsymbol}{$\blacksquare$}% Optional qed square
\makeatother 

% Defines the theorem text style for each type of theorem to one of the three styles above
\newcounter{dummy} 
\numberwithin{dummy}{section}
\theoremstyle{ocrenumbox}
\newtheorem{theoremeT}[dummy]{Theorem}
\newtheorem{problem}{Problem}[chapter]
\newtheorem{exerciseT}{Exercise}[chapter]
\theoremstyle{blacknumex}
\newtheorem{exampleT}{Example}[chapter]
\theoremstyle{blacknumbox}
\newtheorem{vocabulary}{Vocabulary}[chapter]
\newtheorem{definitionT}{Definition}[section]
\newtheorem{corollaryT}[dummy]{Corollary}
\theoremstyle{ocrenum}
\newtheorem{proposition}[dummy]{Proposition}

%----------------------------------------------------------------------------------------
%	DEFINITION OF COLORED BOXES
%----------------------------------------------------------------------------------------

\RequirePackage[framemethod=default]{mdframed} % Required for creating the theorem, definition, exercise and corollary boxes

% Theorem box
\newmdenv[skipabove=7pt,
skipbelow=7pt,
backgroundcolor=black!5,
linecolor=ocre,
innerleftmargin=5pt,
innerrightmargin=5pt,
innertopmargin=5pt,
leftmargin=0cm,
rightmargin=0cm,
innerbottommargin=5pt]{tBox}

% Exercise box	  
\newmdenv[skipabove=7pt,
skipbelow=7pt,
rightline=false,
leftline=true,
topline=false,
bottomline=false,
backgroundcolor=ocre!10,
linecolor=ocre,
innerleftmargin=5pt,
innerrightmargin=5pt,
innertopmargin=5pt,
innerbottommargin=5pt,
leftmargin=0cm,
rightmargin=0cm,
linewidth=4pt]{eBox}	

% Definition box
\newmdenv[skipabove=7pt,
skipbelow=7pt,
rightline=false,
leftline=true,
topline=false,
bottomline=false,
linecolor=ocre,
innerleftmargin=5pt,
innerrightmargin=5pt,
innertopmargin=0pt,
leftmargin=0cm,
rightmargin=0cm,
linewidth=4pt,
innerbottommargin=0pt]{dBox}	

% Corollary box
\newmdenv[skipabove=7pt,
skipbelow=7pt,
rightline=false,
leftline=true,
topline=false,
bottomline=false,
linecolor=gray,
backgroundcolor=black!5,
innerleftmargin=5pt,
innerrightmargin=5pt,
innertopmargin=5pt,
leftmargin=0cm,
rightmargin=0cm,
linewidth=4pt,
innerbottommargin=5pt]{cBox}

% Creates an environment for each type of theorem and assigns it a theorem text style from the "Theorem Styles" section above and a colored box from above
\newenvironment{theorem}{\begin{tBox}\begin{theoremeT}}{\end{theoremeT}\end{tBox}}
\newenvironment{exercise}{\begin{eBox}\begin{exerciseT}}{\hfill{\color{ocre}\tiny\ensuremath{\blacksquare}}\end{exerciseT}\end{eBox}}				  
\newenvironment{definition}{\begin{dBox}\begin{definitionT}}{\end{definitionT}\end{dBox}}	
\newenvironment{example}{\begin{exampleT}}{\hfill{\tiny\ensuremath{\blacksquare}}\end{exampleT}}		
\newenvironment{corollary}{\begin{cBox}\begin{corollaryT}}{\end{corollaryT}\end{cBox}}	

%----------------------------------------------------------------------------------------
%	REMARK ENVIRONMENT
%----------------------------------------------------------------------------------------

\newenvironment{remark}{\par\vspace{10pt}\small % Vertical white space above the remark and smaller font size
	\begin{list}{}{
			\leftmargin=35pt % Indentation on the left
			\rightmargin=25pt}\item\ignorespaces % Indentation on the right
		\makebox[-2.5pt]{\begin{tikzpicture}[overlay]
			\node[draw=ocre!60,line width=1pt,circle,fill=ocre!25,font=\sffamily\bfseries,inner sep=2pt,outer sep=0pt] at (-15pt,0pt){\textcolor{ocre}{R}};\end{tikzpicture}} % Orange R in a circle
		\advance\baselineskip -1pt}{\end{list}\vskip5pt} % Tighter line spacing and white space after remark

%----------------------------------------------------------------------------------------
%	SECTION NUMBERING IN THE MARGIN
%----------------------------------------------------------------------------------------

\makeatletter
\renewcommand{\@seccntformat}[1]{\llap{\textcolor{ocre}{\csname the#1\endcsname}\hspace{1em}}}                    
\renewcommand{\section}{\@startsection{section}{1}{\z@}
	{-4ex \@plus -1ex \@minus -.4ex}
	{1ex \@plus.2ex }
	{\normalfont\large\sffamily\bfseries}}
\renewcommand{\subsection}{\@startsection {subsection}{2}{\z@}
	{-3ex \@plus -0.1ex \@minus -.4ex}
	{0.5ex \@plus.2ex }
	{\normalfont\sffamily\bfseries}}
\renewcommand{\subsubsection}{\@startsection {subsubsection}{3}{\z@}
	{-2ex \@plus -0.1ex \@minus -.2ex}
	{.2ex \@plus.2ex }
	{\normalfont\small\sffamily\bfseries}}                        
\renewcommand\paragraph{\@startsection{paragraph}{4}{\z@}
	{-2ex \@plus-.2ex \@minus .2ex}
	{.1ex}
	{\normalfont\small\sffamily\bfseries}}

%----------------------------------------------------------------------------------------
%	PART HEADINGS
%----------------------------------------------------------------------------------------

% numbered part in the table of contents
\newcommand{\@mypartnumtocformat}[2]{%
	\setlength\fboxsep{0pt}%
	\noindent\colorbox{ocre!20}{\strut\parbox[c][.7cm]{\ecart}{\color{ocre!70}\Large\sffamily\bfseries\centering#1}}\hskip\esp\colorbox{ocre!40}{\strut\parbox[c][.7cm]{\linewidth-\ecart-\esp}{\Large\sffamily\centering#2}}}%
%%%%%%%%%%%%%%%%%%%%%%%%%%%%%%%%%%
% unnumbered part in the table of contents
\newcommand{\@myparttocformat}[1]{%
	\setlength\fboxsep{0pt}%
	\noindent\colorbox{ocre!40}{\strut\parbox[c][.7cm]{\linewidth}{\Large\sffamily\centering#1}}}%
%%%%%%%%%%%%%%%%%%%%%%%%%%%%%%%%%%
\newlength\esp
\setlength\esp{4pt}
\newlength\ecart
\setlength\ecart{1.2cm-\esp}
\newcommand{\thepartimage}{}%
\newcommand{\partimage}[1]{\renewcommand{\thepartimage}{#1}}%
\def\@part[#1]#2{%
	\ifnum \c@secnumdepth >-2\relax%
	\refstepcounter{part}%
	\addcontentsline{toc}{part}{\texorpdfstring{\protect\@mypartnumtocformat{\thepart}{#1}}{\partname~\thepart\ ---\ #1}}
	\else%
	\addcontentsline{toc}{part}{\texorpdfstring{\protect\@myparttocformat{#1}}{#1}}%
	\fi%
	\startcontents%
	\markboth{}{}%
	{\thispagestyle{empty}%
		\begin{tikzpicture}[remember picture,overlay]%
		\node at (current page.north west){\begin{tikzpicture}[remember picture,overlay]%	
			\fill[ocre!20](0cm,0cm) rectangle (\paperwidth,-\paperheight);
			\node[anchor=north] at (4cm,-3.25cm){\color{ocre!40}\fontsize{220}{100}\sffamily\bfseries\@Roman\c@part}; 
			\node[anchor=south east] at (\paperwidth-1cm,-\paperheight+1cm){\parbox[t][][t]{8.5cm}{
					\printcontents{l}{0}{\setcounter{tocdepth}{1}}%
			}};
			\node[anchor=north east] at (\paperwidth-1.5cm,-3.25cm){\parbox[t][][t]{15cm}{\strut\raggedleft\color{white}\fontsize{30}{30}\sffamily\bfseries#2}};
			\end{tikzpicture}};
\end{tikzpicture}}%
\@endpart}
\def\@spart#1{%
\startcontents%
\phantomsection
{\thispagestyle{empty}%
	\begin{tikzpicture}[remember picture,overlay]%
	\node at (current page.north west){\begin{tikzpicture}[remember picture,overlay]%	
		\fill[ocre!20](0cm,0cm) rectangle (\paperwidth,-\paperheight);
		\node[anchor=north east] at (\paperwidth-1.5cm,-3.25cm){\parbox[t][][t]{15cm}{\strut\raggedleft\color{white}\fontsize{30}{30}\sffamily\bfseries#1}};
		\end{tikzpicture}};
\end{tikzpicture}}
\addcontentsline{toc}{part}{\texorpdfstring{%
	\setlength\fboxsep{0pt}%
	\noindent\protect\colorbox{ocre!40}{\strut\protect\parbox[c][.7cm]{\linewidth}{\Large\sffamily\protect\centering #1\quad\mbox{}}}}{#1}}%
\@endpart}
\def\@endpart{\vfil\newpage
\if@twoside
\if@openright
\null
\thispagestyle{empty}%
\newpage
\fi
\fi
\if@tempswa
\twocolumn
\fi}

%----------------------------------------------------------------------------------------
%	CHAPTER HEADINGS
%----------------------------------------------------------------------------------------

% A switch to conditionally include a picture, implemented by  Christian Hupfer
\newif\ifusechapterimage
\usechapterimagetrue
\newcommand{\thechapterimage}{}%
\newcommand{\chapterimage}[1]{\ifusechapterimage\renewcommand{\thechapterimage}{#1}\fi}%
\def\@makechapterhead#1{%
{\parindent \z@ \raggedright \normalfont
\ifnum \c@secnumdepth >\m@ne
\if@mainmatter
\begin{tikzpicture}[remember picture,overlay]
\node at (current page.north west)
{\begin{tikzpicture}[remember picture,overlay]
	\node[anchor=north west,inner sep=0pt] at (0,0) {\ifusechapterimage\includegraphics[width=\paperwidth]{\thechapterimage}\fi};
	\draw[anchor=west] (\Gm@lmargin,-9cm) node [line width=2pt,rounded corners=15pt,draw=ocre,fill=white,fill opacity=0.5,inner sep=15pt]{\strut\makebox[22cm]{}};
	\draw[anchor=west] (\Gm@lmargin+.3cm,-9cm) node {\huge\sffamily\bfseries\color{black}\thechapter. #1\strut};
	\end{tikzpicture}};
\end{tikzpicture}
\else
\begin{tikzpicture}[remember picture,overlay]
\node at (current page.north west)
{\begin{tikzpicture}[remember picture,overlay]
\node[anchor=north west,inner sep=0pt] at (0,0) {\ifusechapterimage\includegraphics[width=\paperwidth]{\thechapterimage}\fi};
\draw[anchor=west] (\Gm@lmargin,-9cm) node [line width=2pt,rounded corners=15pt,draw=ocre,fill=white,fill opacity=0.5,inner sep=15pt]{\strut\makebox[22cm]{}};
\draw[anchor=west] (\Gm@lmargin+.3cm,-9cm) node {\huge\sffamily\bfseries\color{black}#1\strut};
\end{tikzpicture}};
\end{tikzpicture}
\fi\fi\par\vspace*{270\p@}}}

%-------------------------------------------

\def\@makeschapterhead#1{%
\begin{tikzpicture}[remember picture,overlay]
\node at (current page.north west)
{\begin{tikzpicture}[remember picture,overlay]
\node[anchor=north west,inner sep=0pt] at (0,0) {\ifusechapterimage\includegraphics[width=\paperwidth]{\thechapterimage}\fi};
\draw[anchor=west] (\Gm@lmargin,-9cm) node [line width=2pt,rounded corners=15pt,draw=ocre,fill=white,fill opacity=0.5,inner sep=15pt]{\strut\makebox[22cm]{}};
\draw[anchor=west] (\Gm@lmargin+.3cm,-9cm) node {\huge\sffamily\bfseries\color{black}#1\strut};
\end{tikzpicture}};
\end{tikzpicture}
\par\vspace*{270\p@}}
\makeatother

%----------------------------------------------------------------------------------------
%	HYPERLINKS IN THE DOCUMENTS
%----------------------------------------------------------------------------------------

\usepackage{hyperref}
\hypersetup{hidelinks,backref=true,pagebackref=true,hyperindex=true,colorlinks=false,breaklinks=true,urlcolor= ocre,bookmarks=true,bookmarksopen=false,pdftitle={Title},pdfauthor={Author}}
\usepackage{bookmark}
\bookmarksetup{
open,
numbered,
addtohook={%
\ifnum\bookmarkget{level}=0 % chapter
\bookmarksetup{bold}%
\fi
\ifnum\bookmarkget{level}=-1 % part
\bookmarksetup{color=ocre,bold}%
\fi
}
}

\addbibresource{Bronvermelding.bib} 

\begin{document}

\begingroup
\thispagestyle{empty}
\begin{tikzpicture}[remember picture,overlay]
\coordinate [below=12cm] (midpoint) at (current page.north);
\node at (current page.north west)
{\begin{tikzpicture}[remember picture,overlay]
\node[anchor=north west,inner sep=0pt] at (0,0) {\includegraphics[width=\paperwidth]{background}}; % Background image
\draw[anchor=north] (midpoint) node [fill=ocre!30!white,fill opacity=0.6,text opacity=1,inner sep=1cm]{\Huge\centering\bfseries\sffamily\parbox[c][][t]{\paperwidth}{\centering 开发记录 \\[15pt] % Book title
{\Large 卷一}\\[20pt] % Subtitle
{\large Dolphin}}}; % Author name
\end{tikzpicture}};
\end{tikzpicture}
\vfill
\endgroup


%----------------------------------------------------------------------------------------
%	BLANK PAGE
%----------------------------------------------------------------------------------------

\newpage
~\vfill
\thispagestyle{empty}

%----------------------------------------------------------------------------------------
%	COPYRIGHT PAGE
%----------------------------------------------------------------------------------------

\newpage
~\vfill
\thispagestyle{empty}

\noindent Copyright \textcopyright\ 2017 Xiaoqiang Jiang\\ % Copyright notice

\noindent \textsc{Edited by Xiaoqiang Jiang}\\ % Publisher

\noindent \textsc{\url{http://jiangxiaoqiang.github.com/}}\\

\noindent All Rights Reserved.\\ % License information

\noindent \textit{Version \currenttime, \today} % Printing/edition date


%----------------------------------------------------------------------------------------
%	TABLE OF CONTENTS
%----------------------------------------------------------------------------------------

%\usechapterimagefalse % If you don't want to include a chapter image, use this to toggle images off - it can be enabled later with \usechapterimagetrue

\chapterimage{chapterhead1.pdf} % Table of contents heading image

\pagestyle{empty} % No headers

\tableofcontents % Print the table of contents itself

\cleardoublepage % Forces the first chapter to start on an odd page so it's on the right

\pagestyle{fancy} % Print headers again

%----------------------------------------------------------------------------------------
%	BLANK PAGE
%----------------------------------------------------------------------------------------


这里记录的是一些比较杂乱的笔记,绝大多数文字皆来源于网络,不是自己的原创,这里没有高深的算法,没有宏伟的技术及系统架构,只是一些平时工作中遇到的一些问题,和解决问题的思路以及所采用的方案。由于平时工作时还没有遇到前人没有遇到过的问题需要自己发明方去解决(其实真的遇到估计也是没辙),所以绝大部分内容是为了避免再遇到同样的问题时,又需要到处去搜寻,索性将之记录下来,以便于下次可以快刀斩乱麻,迅速解决问题。


\mainmatter

%----------------------------------------------------------------------------------------
% PART
%----------------------------------------------------------------------------------------

\part{Tool}

\section{Tool Set}

\subsection{ECS(Elastic Compute Service)}

配置内网ECS端口映射规则,在云服务器ECS-->网络和安全-->安全组-->配置规则.

\subsection{shadowsocks}

启动ss服务器:

\begin{lstlisting}[language=Bash]
#可以查看启动日志
ssserver -c /home/ec2-user/shadowsocks.json
#后台启动
ssserver -c /home/ec2-user/shadowsocks.json -d start
\end{lstlisting}

Shadowsocks客户端操作:

\begin{lstlisting}[language=Bash]
sudo apt-get install python-pip
sudo apt-get install python-setuptools m2crypto
#安装Shadowsocks(Ubuntu/Fedora)
pip install shadowsocks
#前台启动
#可以看到实时的日志输出
#关闭终端后代理断开
sslocal -c /etc/shadowsocks/shadowsocks.json
#后台启动
sslocal -c /etc/shadowsocks/shadowsocks.json -d start
\end{lstlisting}

\subsection{youtube-dl}

YouTube视频一般是不能下载的,但是是国内访问YouTube比较慢,经常卡顿,所以可以使用youtube-dl工具下载YouTube视频:

\begin{lstlisting}[language=Bash]
#下载默认的视频格式
youtube-dl https://www.youtube.com/watch?v=SnHxKQiXrFU
#查看所有视频格式
youtube-dl -F https://www.youtube.com/watch?v=SnHxKQiXrFU
#下载指定清晰度的视频
#137为指定视频格式的编码format code
youtube-dl -f 137 https://www.youtube.com/watch?v=SnHxKQiXrFU
\end{lstlisting}

查看YouTube所有格式视频输出效果如图所示:

\begin{figure}[htbp]
	\centering
	\includegraphics[scale=0.4]{checkyoubevideoformat.png}
	\caption{查看所有格式视频}
	\label{fig:checkyoubevideoformat}
\end{figure}




%----------------------------------------------------------------------------------------
%	CHAPTER 1
%----------------------------------------------------------------------------------------

\chapterimage{chapterhead2.pdf} % Chapter heading image


\subsection{Jenkins}



\chapter{OpenVPN}

OpenVPN从2001年开始开发,使用的是C语言。此处使用的OpenVPN版本是2.4.1。如果使用Mac下的brew工具安装,则OpenVPN目录在:/usr/local/Cellar/openvpn/2.4.1,OpenVPN的配置文件在:/usr/local/etc/openvpn。目前OpenVPN能在Solaris、Linux、OpenBSD、FreeBSD、NetBSD、Mac OS X与Microsoft Windows以及Android和iOS上运行,并包含了许多安全性的功能。此处的服务器使用的是CentOS 7.3,客户端包含Fedora 24、Ubuntu 14.04、Ubuntu 16.04、Window 7、Windows 10。在OpenVPN网络中查看存活的主机:

\begin{lstlisting}[language=bash]
nmap -A -T4 10.0.0.*
\end{lstlisting}

\subsection{安装(Install)}

安装基础包:

\begin{lstlisting}[language=bash]
sudo yum -y install openssl openssl-devel lzo openvpn easy-rsa --allowerasing
#手动安装
wget -c https://swupdate.openvpn.org/community/releases/openvpn-2.4.1.tar.gz
tar -zxvf openvpn-2.4.1.tar.gz
./configure
make && make install
\end{lstlisting}

LZO 是致力于解压速度的一种数据压缩算法,LZO 是 Lempel-Ziv-Oberhumer 的缩写。这个算法是无损算法,参考实现程序是线程安全的。 实现它的一个自由软件工具是lzop。最初的库是用 ANSI C 编写、并且遵从 GNU通用公共许可证发布的。现在 LZO 有用于 Perl、Python 以及 Java 的各种版本。代码版权的所有者是 Markus F. X. J. Oberhumer。

\subsection{生成客户端证书}

生成客户端证书端步骤如下:

\begin{lstlisting}[language=Bash]
#建立openvpn配置文件存放目录
mkdir -p /etc/openvpn
#如果没有安装easy-rsa工具
yum install easy-rsa
#建立一个空的pki结构,生成一系列的文件和目录
./easyrsa init-pki
\end{lstlisting}

PKI:Public Key Infrastructure公钥基础设施。生成请求:

\begin{lstlisting}[language=Bash]
./easyrsa gen-req dolphinfedora
\end{lstlisting}

输入PEM验证码。PEM - Privacy Enhanced Mail,打开看文本格式,以"-----BEGIN..."开头, "-----END..."结尾,内容是BASE64编码。
Apache和*NIX服务器偏向于使用这种编码格式.签约:

\begin{lstlisting}[language=Bash]
#切换到服务端生成rsa的目录
#导入req
./easyrsa import-req ~/client/easyrsa/easy-rsa-master/easyrsa3/pki/reqs/dolphinfedora.req dolphinfedora
#用户签约,根据提示输入服务端的ca密码
./easyrsa sign client dolphinfedora
\end{lstlisting}

PKI:Public Key Infrastructure公钥基础设施。输入PEM验证码。PEM - Privacy Enhanced Mail,打开看文本格式,以"-----BEGIN..."开头, "-----END..."结尾,内容是BASE64编码.查看PEM格式证书的信息:openssl x509 -in certificate.pem -text -noout。Apache和*NIX服务器偏向于使用这种编码格式.服务端生成的文件有:

\begin{tabular}{|c|p{5cm}|c|}
	\hline
	\multirow{1}{*}{文件名称}
	& \multicolumn{1}{c|}{说明(Purpose)} 
	& \multicolumn{1}{c|}{位置} \\			
	\cline{1-3}
	ca.crt  & 根证书(Root CA certificate)件 & Server+All Clients	\\
	\hline
	reqs/server.req  & &\\
	\hline
	reqs/dolphin.req  & &\\
	\hline
	private/ca.key & 根证书私钥(Root CA key) & key signing machine only\\
	\hline
	private/server.key && \\
	\hline
	issued/server.crt & 服务器证书Server Certificate & server only\\
	\hline
	issued/dolphin.crt && \\
	\hline
	dh.pem & Diffie Hellman parameters & server only \\
	\hline
\end{tabular}

客户端生成的文件有:

\begin{tabular}{|c|p{8cm}|c|}
	\hline
	\multirow{1}{*}{序号}
	& \multicolumn{1}{c|}{名称}  \\			
	\cline{1-2}
	private/dolphinclient.key  & \\
	\hline
	reqs/sdolphinclient.req & \\
	\hline
\end{tabular}

拷贝出客户端证书文件:

\begin{lstlisting}[language=Bash]
cp easyrsa/easy-rsa-master/easyrsa3/pki/ca.crt ~/dolphinfedora/
cp easyrsa/easy-rsa-master/easyrsa3/pki/issued/dolphinfedora.crt ~/dolphinfedora/
cp ~/client/easyrsa/easy-rsa-master/easyrsa3/pki/private/dolphinfedora.key ~/dolphinfedora/
\end{lstlisting}


启动OpenVPN:

\begin{lstlisting}[language=Bash]
sudo openvpn server.conf
# Mac下启动OpenVPN
sudo /usr/local/Cellar/openvpn/2.4.1/sbin/openvpn /usr/local/etc/openvpn/client.conf
# 需要以后台交互方式启动时
screen sudo openvpn client.conf
\end{lstlisting}

客户端端配置如下:

\begin{lstlisting}[language=Bash]
client         #指定当前VPN是客户端
dev tun        #必须与服务器端的保持一致
proto udp      #必须与服务器端的保持一致
#指定连接的远程服务器的实际IP地址和端口号
remote 192.168.1.106 1194      
#断线自动重新连接
#在网络不稳定的情况下(例如:笔记本电脑无线网络)非常有用
resolv-retry infinite
nobind         #不绑定特定的本地端口号
persist-key
persist-tun
ca ca.crt      #指定CA证书的文件路径
cert client1.crt       #指定当前客户端的证书文件路径
key client1.key    #指定当前客户端的私钥文件路径
ns-cert-type server      #指定采用服务器校验方式
#如果服务器设置了防御DoS等攻击的ta.key
#则必须每个客户端开启;如果未设置,则注释掉这一行;
tls-auth ta.key 1     
comp-lzo              #与服务器保持一致
#指定日志文件的记录详细级别,可选0-9,等级越高日志内容越详细
verb 3                
\end{lstlisting}

配置ns-cert-type(Netscape Cert Type)指定为server主要是防止中间人攻击(Man-in-the-Middle Attack)。在服务端做如下配置:

\begin{lstlisting}[language=Bash]
nsCertType server
\end{lstlisting}


\chapter*{Bibliography}
\addcontentsline{toc}{chapter}{\textcolor{ocre}{Bibliography}}
\section*{Books}
\addcontentsline{toc}{section}{Books}
\printbibliography[heading=bibempty,type=book]
\section*{Articles}
\addcontentsline{toc}{section}{Articles}
\printbibliography[heading=bibempty,type=article]

%----------------------------------------------------------------------------------------
%	INDEX
%----------------------------------------------------------------------------------------

\cleardoublepage
\phantomsection
\setlength{\columnsep}{0.75cm}
\addcontentsline{toc}{chapter}{\textcolor{ocre}{Index}}
\printindex

%----------------------------------------------------------------------------------------

\end{document}
